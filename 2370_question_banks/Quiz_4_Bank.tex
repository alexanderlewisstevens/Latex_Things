% Streamlined LaTeX template for question banks
\documentclass[12pt]{article}
\usepackage[margin=1in]{geometry}
\usepackage{enumitem}
\usepackage{parskip}
\usepackage{amsmath, amssymb}
\usepackage{listings}
\usepackage{fancyhdr}
\pagestyle{fancy}

% Header strip for compact info
\lhead{CS/Discrete Math}
\chead{Quiz 4 Practice Questions}
\rhead{Practice-Problem-Bank}
\cfoot{\thepage}

\begin{document}

\section*{Problems}

\subsection*{Quicksort and Randomized Quicksort}
\begin{itemize}[leftmargin=1.5em]
  \item Describe how Quicksort partitions an array. What happens during each recursive step?
  \item For the input array $[6, 5, 4, 1, 2, 3]$, trace the calls made by Quicksort when the last element is always chosen as pivot. Write the contents of the array after each step.
  \item What is the best-case scenario for Quicksort, and what input pattern produces it?
  \item What is the worst-case scenario for Quicksort, and what input pattern produces it?
  \item Write the recurrence for Quicksort's runtime in the best-case (perfectly balanced splits) and solve it using the Master Theorem.
  \item Write the recurrence for Quicksort's runtime in the worst-case (one element per partition) and solve it.
\end{itemize}

\subsection*{Convex Hull}
\begin{itemize}[leftmargin=1.5em]
  \item Define the Convex Hull problem by stating what the input and output should represent.
  \item Describe the brute-force approach to computing the convex hull of a set of 2D points. What is its runtime and why?
  \item Describe the high-level idea behind the QuickHull algorithm.
  \item What is the worst-case time complexity of QuickHull? Draw a point configuration where it occurs.
  \item Describe the recursive structure of QuickHull and write its recurrence when splits are even.
\end{itemize}

\subsection*{Writing and Solving Recurrences}
\begin{itemize}[leftmargin=1.5em]
  \item Write a recurrence relation for the following pseudocode:
\begin{lstlisting}
def mystery(n):
    if n <= 1:
        return 1
    return 3 * mystery(n//2) + n
\end{lstlisting}
  \item Solve the recurrence from the previous problem using the Master Theorem.
  \item Use the recursion tree method to solve: $T(n) = T(n/3) + n$
  \item Solve $T(n) = 2T(n/2) + n$ using the Master Theorem.
  \item Solve $T(n) = T(n-1) + n$ using the recursion tree method.
  \item Given a recurrence $T(n) = 4T(n/2) + n^2$, classify the function using the Master Theorem.
\end{itemize}

\end{document}