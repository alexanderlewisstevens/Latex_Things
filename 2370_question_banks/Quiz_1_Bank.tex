% Streamlined LaTeX template for question banks
\documentclass[12pt]{article}
\usepackage[margin=1in]{geometry}
\usepackage{enumitem}
\usepackage{parskip}
\usepackage{amsmath, amssymb}
\usepackage{listings}
\usepackage{fancyhdr}
\pagestyle{fancy}

% Header strip for compact info
\lhead{CS/Discrete Math}
\chead{Quiz 1 Practice Questions}
\rhead{Practice-Problem-Bank}
\cfoot{\thepage}

\begin{document}

\section*{Concepts}
\begin{itemize}[leftmargin=1.5em]
  \item Brute Force Algorithms: Practice solving the algorithms below using brute force. They don't need to be correct or efficient.
  \item Understand the basics of the three sorting algorithms:
  \begin{itemize}
    \item Be ready to apply the algorithms to an input
    \item Understand what input gives the best and worst case for each algorithm
    \item Understand the procedures each algorithm uses and be ready to write pseudocode
  \end{itemize}
\end{itemize}

\section*{Problems}

\subsection*{1. Two Sum}
Given an array of integers \texttt{nums} and an integer \texttt{target}, return the indices of the two numbers such that they add up to \texttt{target}. Each input has exactly one solution, and you may not use the same element twice.

\textbf{Example:}
\begin{verbatim}
Input: nums = [2,7,11,15], target = 9
Output: [0,1]
\end{verbatim}

\subsection*{2. Best Time to Buy and Sell Stock}
Given an array \texttt{prices} where \texttt{prices[i]} is the price of a stock on the \texttt{i-th} day, find the maximum profit from a single buy and sell operation. You must buy before you sell.

\textbf{Example:}
\begin{verbatim}
Input: prices = [7,1,5,3,6,4]
Output: 5 (buy at 1, sell at 6)
\end{verbatim}

\subsection*{3. Contains Duplicate}
Given an array of integers, check if any value appears at least twice.

\textbf{Example:}
\begin{verbatim}
Input: [1, 2, 3, 1]
Output: True
\end{verbatim}

\subsection*{4. Maximum Subarray}
Given an integer array \texttt{nums}, find the contiguous subarray (containing at least one number) with the largest sum.

\textbf{Example:}
\begin{verbatim}
Input: [-2,1,-3,4,-1,2,1,-5,4]
Output: 6 ([4,-1,2,1])
\end{verbatim}

\subsection*{5. Longest Common Prefix}
Write a function to find the longest common prefix string amongst an array of strings. If there is no common prefix, return an empty string \texttt{""}.

\textbf{Example:}
\begin{verbatim}
Input: ["flower","flow","flight"]
Output: "fl"
\end{verbatim}

\subsection*{6. Valid Palindrome}
Given a string, determine if it is a palindrome considering only alphanumeric characters and ignoring cases.

\textbf{Example:}
\begin{verbatim}
Input: "A man, a plan, a canal: Panama"
Output: True
\end{verbatim}

\subsection*{7. First Unique Character in a String}
Given a string, find the first non-repeating character and return its index. If it does not exist, return -1.

\textbf{Example:}
\begin{verbatim}
Input: "leetcode"
Output: 0
\end{verbatim}

\subsection*{8. Intersection of Two Arrays}
Given two arrays, return their intersection (unique elements only).

\textbf{Example:}
\begin{verbatim}
Input: nums1 = [1,2,2,1], nums2 = [2,2]
Output: [2]
\end{verbatim}

\end{document}
